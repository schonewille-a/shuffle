\documentclass[12pt, a4paper]{article}

%% Language and font encodings
\usepackage[english]{babel}
\usepackage[utf8x]{inputenc}
\usepackage[T1]{fontenc}

%% Sets page size and margins
\usepackage[a4paper,top=3cm,bottom=2cm,left=3cm,right=3cm,marginparwidth=1.75cm]{geometry}

%% Useful packages
\usepackage{amsmath}
\usepackage{graphicx}
\usepackage[colorinlistoftodos]{todonotes}
\usepackage[colorlinks=true, allcolors=blue]{hyperref}

\title{}
\title{%
  SFWR 3XA3: Problem Statement \\
  \hfill \break
  \large ASK Studios, Group 16 \\
    L02}
\begin{document}
\maketitle
\section{Problem Statement}
\subsection{The problem we are trying to solve}
Many people can't justify paying money to subscribe to a music streaming service such as Spotify, Google Play Music, or Apple Music. For this reason, music lovers resort to YouTube as a free alternative for listening to music. Although YouTube is a great alternative, it lacks several key features that its music streaming competitors incorporate. This poses as a large deterrent to most music seekers as these features are what make the overall experience more enjoyable. Some of these features include a radio function for recommending music, browsing by genre, excluding and including specific artists, and neatly organizing saved songs.
\subsection{Why is the problem important?}
Many people enjoy music as entertainment or as a way to focus when working or studying and music streaming makes this very easy. In the past the only way the listen to your music was purchasing it and storing it locally, which is expensive and takes up valuable space.  With streaming becoming more and more relevant, services like Spotify began, but Spotify and similar services use an expensive subscription payment model.  This leads to many listeners either spending large amounts of money on purchasing music, or resorting to piracy, which does not net any money for the artist.
\subsection{What is the context of the problem you are solving?}
The stakeholders in this project are group members, future developers of this project or Shuffle, music listeners, YouTube developers, music labels, and music artists.  Shuffle is a web application backed by HTML5, CSS and JavaScript, which are supported by most web environments on most popular operating systems.  The scope of this project is to take the existing project, Shuffle, and redesign it, with proper design and documentation processes.  ASK Studios will maintain the product throughout its lifespan.





\end{document}